\begin{solution}\ \\
    若存在$M$为$\mathbb{F}_3[x]$的理想且$M$真包含$I$的情况下(即$M\supsetneqq I$),则必定存在一个不属于$I$的多项式$g(x)\in M\backslash I$,使得$f(x)\nmid g(x)$。\\
    因为$f(x)$为不可约多项式,所以有$(f(x),g(x))=1$,由广义欧几里得除法以及广义Bézout定理可知:
    \begin{equation*}
        \exists s(x),t(x)\in\mathbb{F}_3[x],s.t.\quad s(x)f(x)+t(x)g(x)=1
    \end{equation*}
    由理想的定义可知,若$f(x),g(x)\in M\ s(x),t(x)\in\mathbb{F}_3[x]$,则$s(x)f(x)+t(x)g(x)=1\in M$,故$M=\mathbb{F}_3[x]$,即$I$与$\mathbb{F}_3[x]$之间不存在中间理想,所以由$f(x)$生成的(主)理想$I=(f(x))$是多项式环$\mathbb{F}_3[3]$中的极大理想。
\end{solution}