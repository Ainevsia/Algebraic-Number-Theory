\begin{solution}\ \\
    \begin{equation*}
        \begin{aligned}
            \alpha&=a_0\beta+a_1\sigma(\beta)+a_2\sigma^2(\beta)+\dots+a_{n-1}\sigma^{n-1}(\beta)\\
            \sigma^d(\alpha)&=\sigma^d(a_0\beta+a_1\sigma(\beta)+a_2\sigma^2(\beta)+\dots+a_{n-1}\sigma^{n-1}(\beta))\\
            &=\sigma^d(a_0\beta)+\sigma^d(a_1\sigma(\beta))+\sigma^d(a_2\sigma^2(\beta))+\dots+\sigma^d(a_{n-1}\sigma^{n-1}(\beta))\\
            &=\sigma^d(a_0)\sigma^d(\beta)+\sigma^d(a_1)\sigma^d(\sigma(\beta))+\sigma^d(a_2)\sigma^d(\sigma^2(\beta))+\dots+\sigma^d(a_{n-1})\sigma^d(\sigma^{n-1}(\beta))\\
            &=a_0\sigma^d(\beta)+a_1\sigma^{d+1}(\beta)+a_2\sigma^{d+2}(\beta)+\dots+a_{n-1}\sigma^{d+n-1}(\beta)
        \end{aligned}
    \end{equation*}
    $\because\sigma^d(\alpha)=\alpha$且$\sigma^n$为恒等映射,对比相应基底的系数可知:排列$(a_0,a_1,a_2,\dots,a_{n-1})$循环右移d位后与原先的排列相同。也就是说:
    \begin{equation*}
        \begin{aligned}
            a_0&=&a_d&=&a_{2d}&=&\dots&=&a_{(\frac{n}{d}-1)d}\\
            a_1&=&a_{d+1}&=&a_{2d+1}&=&\dots&=&a_{(\frac{n}{d}-1)d+1}\\
            \ldots\\
            a_{d-1}&=&a_{2d-1}&=&a_{3d-1}&=&\dots&=&a_{n-1}\\
        \end{aligned}
    \end{equation*}
    回到题目的问题,系数$a_0,a_1,a_2,\dots,a_{n-1}$之间的关系就是每隔d项的系数相等,独立的系数只有$d-1$个,将系数相同的基底合并起来可以写成如下的抽象的形式:
    \begin{equation*}
        \begin{aligned}
            \textbf{I}(\sigma_q^d)&=\{\alpha|\sigma^d(\alpha)=\alpha\}\\
            &=\{a_0\gamma+a_1\sigma(\gamma)+a_2\sigma^2(\gamma)+\dots+a_{d-1}\sigma^{d-1}(\gamma)|\gamma=\beta+\sigma^d(\beta)+\sigma^{2d}(\beta)+\dots\\+\sigma^{(\frac{n}{d}-1)d}(\beta)\}
        \end{aligned}
    \end{equation*}
    其中$a_0$到$a_{d-1}$是独立的系数个数,$\gamma$为相同系数的基底合并后的简写。
    \begin{tcolorbox}
        \textbf{具体来说} 
        \textit{取$n=8, d=2, d|n$, 所以$(a_0,a_1,a_2,a_3,a_4,a_5,a_6,a_7)$循环右移$2$位后为}$(a_6,a_7,a_0,a_1,a_2,a_3,a_4,a_5)$,与原先相等,即
        \begin{equation*}
            \begin{aligned}
                &(a_0,a_1,a_2,a_3,a_4,a_5,a_6,a_7)\\
                =&(a_6,a_7,a_0,a_1,a_2,a_3,a_4,a_5)
            \end{aligned}
        \end{equation*}
        所以$a_0=a_2=a_4=a_6,a_1=a_3=a_5=a_7$,$\alpha$可表示为
        \begin{equation*}
            \begin{aligned}
                \textbf{I}(\sigma_q^d)&=\{\alpha|\sigma^d(\alpha)=\alpha\}\\
                &=\{a_0\gamma+a_1\sigma(\gamma)|\gamma=\beta+\sigma^{2}(\beta)+\sigma^{4}(\beta)+\sigma^{6}(\beta)\}
            \end{aligned}
        \end{equation*}
    \end{tcolorbox}
\end{solution}