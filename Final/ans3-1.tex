\begin{tcolorbox}
    \textbf{知识点} 
    \textit{素域及域的特征p272-273,自同态、自同构,二项展开,映射}
\end{tcolorbox}
\begin{solution}\ \\
    取任意的$\alpha, \beta\in\mathbb{F}_{q^n}$,有
    \begin{equation*}
        \sigma_q(\alpha+\beta)=(\alpha+\beta)^q=\alpha^q+\beta^q+\sum\limits_{k=1}^{q-1}\frac{q!}{k!(q-k)!}\alpha^k\beta^{q-k}
    \end{equation*}
    因为域$\mathbb{F}_{q}=\mathbb{F}_{p^m}$由域$\mathbb{F}_{p}$扩张得到,而域$\mathbb{F}_{q^n}$由域$\mathbb{F}_{q}$扩张得到,$p$为素数,所以域$\mathbb{F}_{p}$为域$\mathbb{F}_{q^n}$的素域,域$\mathbb{F}_{q^n}$的特征$char(\mathbb{F}_{q^n})=p$,由域的特征的定义可知对$\forall\gamma\in\mathbb{F}_{q^n}$:
    \begin{equation*}
        \begin{aligned}
            p\cdot\gamma&=0\\ \Rightarrow q\cdot\gamma&=p^m\cdot\gamma=0
        \end{aligned}
    \end{equation*}
    $\because1\leq k\leq q-1,\ (q, k!(q-k)!)=1,\ \therefore q\sum\limits_{k=1}^{q-1}\frac{(q-1)!}{k!(q-k)!}\alpha^k\beta^{q-k}=0$,即
    \begin{equation*}
        \sigma_q(\alpha+\beta)=\alpha^q+\beta^q+q\sum\limits_{k=1}^{q-1}\frac{(q-1)!}{k!(q-k)!}\alpha^k\beta^{q-k}=\alpha^q+\beta^q=\sigma_q(\alpha)+\sigma_q(\beta)
    \end{equation*}
    所以映射$\sigma_q$保持加法。又因为
    \begin{equation*}
        \sigma_q(\alpha\beta)=(\alpha\beta)^q=\alpha^q\beta^q=\sigma_q(\alpha)\sigma_q(\beta)
    \end{equation*}
    所以映射$\sigma_q$保持乘法。由同态的定义可知,$\sigma_q$是$\mathbb{F}_{q^n}$的自同态,要证明$\sigma_q$是$\mathbb{F}_{q^n}$的自同构,只需证明$\sigma_q$为一一映射即可:
    \begin{enumerate}
        \item 先证明$\sigma_q$为单射,即证$ker(\sigma_q)=\{0\}$,其中0为有限域$\mathbb{F}_{q^n}$的加法零元,其中$ker(\sigma_q)=\{a\in\mathbb{F}_{q^n}|\sigma_q(a)=0\}$,即被$\sigma_q$映射至加法零元的元素集合。\\
        $\because\sigma_q(a)=a^q=0\Rightarrow a=0,\ \therefore ker(\sigma_q)={0}$,所以$\sigma_q$为单射。
        \item 再证$\sigma_q$为满射,由于$\mathbb{F}_{q^n}$为有限域,只含有有限个元素,又$\sigma_q$是$\mathbb{F}_{q^n}$自身到自身的映射,故$\sigma_q$单射必须是满的。
    \end{enumerate}
    所以$\sigma_q$为$\mathbb{F}_{q^n}$的自同构。因为有限域$\mathbb{F}_{q}$中一共只有q个元素,其中元素的指数是q的因数,所以对$\forall a\in\mathbb{F}_{q},\ \sigma_q(a)=a^q=a$,所以$\sigma_q$为$\mathbb{F}_{q^n}$的$\mathbb{F}_{q}$-自同构,不动元是$\mathbb{F}_{q}$。
\end{solution}